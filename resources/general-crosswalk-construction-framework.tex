\documentclass{article}
\usepackage[utf8]{inputenc}
\usepackage[left=2.50cm, right=2.50cm, top=2.50cm, bottom=2.50cm]{geometry}
\usepackage{enumitem}
\usepackage{bm}
\usepackage{hyperref}
\hypersetup{
    colorlinks=true,
    linkcolor=blue,
    filecolor=magenta,
    urlcolor=cyan,
}

\title{General Crosswalk Construction Framework}
\author{Jonathon Schroeder and James Gaboardi}
\date{May 2020}

\begin{document}

\maketitle

\section{Notation formatting}
\begin{itemize}
  \item {\bf Bold} = variable or set
  \item \textit{Italic} = a single instance (= item in set)
  \item Non-italic = a set
  \item UPPER CASE = input parameter
  \item lower case = derived from input parameters
\end{itemize}

\section{Goal specification}
Generate a crosswalk $\bm{X_{\mathbf{ST}}}$ ...
\begin{itemize}
  \item From source zones $\mathbf{S}$ (geographic level $\bm{G_{\mathbf{S}}}$ in year $\bm{Y_{\mathbf{S}}}$)
  \item To target zones $\mathbf{T}$ (geographic level $\bm{G_{\mathbf{T}}}$ in year $\bm{Y_{\mathbf{T}}}$)
  \item Including exactly one record per atom $\bm{st}$ (an intersection between source zone $\bm{s}$ and target zone $\bm{t}$)
  \item With interpolation weights $\bm{w_{\mathbf{ST}}}$
    \begin{itemize}
    \item A single weight $\bm{w_{\mathbf{cst}}}$ for each count variable $\bm{c}$ in $\mathbf{C}$, for each atom $\bm{st}$
      \begin{itemize}
      \item $\bm{w_{\mathbf{cst}}} =$ proportion of $\bm{c}$ in $\bm{s}$ (denominator) that is also in $\bm{st}$ (numerator) $= \displaystyle \frac{\bm{c_{st}}}{\bm{c_{s}}}$
      \item All $\mathbf{C}$ are count variables (e.g., population, housing units, etc.) that have been reported for a set of sub-zones $\mathbf{S^\prime}$ (blocks)
      \end{itemize}
    \end{itemize}
  \item Build from an existing crosswalk $\bm{X_{\mathbf{S^\prime T^\prime}}}$ ...
    \begin{itemize}
    \item From source sub-zones $\mathbf{S^\prime}$, which nest within $\mathbf{S}$
    \item To source sub-zones $\mathbf{T^\prime}$, which nest within $\mathbf{T}$
    \item In our setting, we can assume:
      \begin{itemize}
      \item $\bm{G_{\mathbf{S^\prime}}} = \bm{G_{\mathbf{T^\prime}}} =$ blocks
      \item $\bm{Y_{\mathbf{S^\prime}}} = \bm{Y_{\mathbf{S}}}$ and $\bm{Y_{\mathbf{T^\prime}}} = \bm{Y_{\mathbf{T}}}$
      \end{itemize}
    \item Includes weights $\bm{w_{\mathbf{S^\prime T^\prime}}}$ indicating proportion of each source sub-zone's features (population \& housing) in each sub-zone atom $\bm{s^\prime t^\prime}$.
    \end{itemize}
  \item Include every $\bm{s}$ in $\mathbf{S}$ and every $\bm{t}$ in $\mathbf{T}$.
    \begin{itemize}
    \item Atom records may have \textit{null} $\bm{s}$ where $\bm{t}$ where a zone in one set lies beyond the spatial extent of the other set, or the intersection is outside the extent of $\bm{X_{\mathbf{S^\prime T^\prime}}}$.
      \begin{itemize}
      \item In our case, $\bm{X_{\mathbf{S^\prime T^\prime}}}$ is a block-to-block crosswalk based on NHGIS shapefiles, which are clipped at the coast. The 1990-2010 crosswalk omits ``off-coast'' 1990 blocks that are not in the shapefile. For crosswalks with 1990 source zones, we may have \textit{null} $\bm{t}$ for source zones that lie entirely off-coast.
      \end{itemize}
    \end{itemize}
\end{itemize}

\section{Summary of key input parameters}
\begin{itemize}
  \item $\bm{G_{\mathbf{S}}} = $ source geographic level
  \item $\bm{Y_{\mathbf{S}}} = $ source year
  \item $\bm{G_{\mathbf{T}}} = $ target geographic level
  \item $\bm{Y_{\mathbf{T}}} = $ target year
  \item $\mathbf{C} = $ set of count variables for which to derive separate weights
\end{itemize}

\section{General steps}

\begin{enumerate}
  \item Obtain \& load sub-zone crosswalk (blocks-to-blocks) $\bm{X_{\mathbf{S^\prime T^\prime}}}$.
  \item Obtain \& load data for source sub-zone counts (source-year block data) $\bm{C_{\mathbf{S^\prime}}}$.
  \begin{enumerate}
    \item Include any identifiers needed to associate $\mathbf{S^\prime}$ with $\mathbf{S}$.
  \end{enumerate}
  \item Join base crosswalk $\bm{X_{\mathbf{S^\prime T^\prime}}}$ to source sub-zone data $\bm{C_{\mathbf{S^\prime}}}$ on $\mathbf{S^\prime}$ identifiers.
    \begin{enumerate}
     \item Use a ``left join'' to ensure that all sub-zone atoms are included, even those without a matching record in the sub-zone data file (especially important for 1990 blocks).
     \end{enumerate}
   \item For each sub-zone atom $\bm{s^\prime t^\prime}$, identify encompassing zones $\bm{s}$ and $\bm{t}$:
  \begin{enumerate}
    \item If possible, derive $\mathbf{S}$ and $\mathbf{T}$ identifiers from $\mathbf{S^\prime}$ and $\mathbf{T^\prime}$ identifiers (e.g., tract ID is in block ID).
    \item Else if possible, derive $\mathbf{S}$ identifiers from source sub-zone data from step 2.
    \item Else, obtain identifiers through other means...
      \begin{enumerate}
      \item 1990 block-group parts require some \href{https://github.com/jGaboardi/nhgisxwalk/blob/master/resources/handling-1990-no-data-blocks-in-crosswalks.pdf}{special handling} because neither 4a nor 4b pertain to all BGPs.
      \item If we generate crosswalks for target zones that cannot be identified from block IDs (e.g., places, county subdivisions, etc.), we'll need to add a step to join block crosswalk to target-year block data that includes identifiers for the target zones.
     \end{enumerate}
   \item Where $\bm{s^\prime}$ is \textit{null} (= ``''), omit these dummy sub-zone atoms from subsequent computations.
      \begin{enumerate}
      \item This may drop some valid $\bm{t}$ from the computations, but step 9 will re-add them if needed.
      \end{enumerate}
   \end{enumerate}
\item Compute counts for all weighting variables in each sub-zone atom: $\bm{c_{\mathbf{S^\prime T^\prime}}} = \bm{w_{\mathbf{S^\prime T^\prime}}} * \bm{C_{\mathbf{S^\prime}}}$.
\item Compute counts for all weighting variables in each atom of interest: $\bm{c_{\mathbf{ST}}} = \sum \bm{c_{\mathbf{S^\prime T^\prime}}}$ group by $\mathbf{S}$, $\mathbf{T}$.
  \begin{enumerate}
  \item Steps 5 \& 6 can be combined into single formula by substituting $\bm{w_{\mathbf{S^\prime T^\prime}}} * \bm{C_{\mathbf{S^\prime}}}$ for $\bm{c_{\mathbf{S^\prime T^\prime}}}$ in step 6.
  \end{enumerate}
\item Compute counts for all weighting variables in each source zone: $\bm{c_{\mathbf{S}}} = \sum \bm{c_{\mathbf{ST}}}$ group by $\mathbf{S}$.
\item Compute all weights for all atoms of interest: $\bm{w_{\mathbf{CST}}} = \displaystyle \frac{\bm{c_{st}}}{\bm{c_{s}}}$.
  \begin{enumerate}
  \item If $\bm{c_{s}} = 0$, set $\bm{w_{\mathbf{cst}}} = 0$.
  \end{enumerate}
\item If $\bm{w_{\mathbf{CST}}}$ is missing data for any $\bm{s}$ in $\mathbf{S}$ or $\bm{t}$ in $\mathbf{T}$, add dummy atoms with \textit{null} $\bm{t}$ for non-\textit{null} $\bm{s}$ or \textit{null} $\bm{s}$ for non-\textit{null} $\bm{t}$, and set $\bm{w_{\mathbf{cst}}}$ = 0.

\item Export clean, complete file for distribution.
  \begin{enumerate}
  \item Exact specifications TBD.
  \end{enumerate}
\end{enumerate}



\end{document}
